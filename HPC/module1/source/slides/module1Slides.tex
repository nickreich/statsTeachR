  

%% preamble document for statsTeachR beamer slides  

\documentclass[table]{beamer}\usepackage[]{graphicx}\usepackage[]{color}
%% maxwidth is the original width if it is less than linewidth
%% otherwise use linewidth (to make sure the graphics do not exceed the margin)
\makeatletter
\def\maxwidth{ %
  \ifdim\Gin@nat@width>\linewidth
    \linewidth
  \else
    \Gin@nat@width
  \fi
}
\makeatother

\definecolor{fgcolor}{rgb}{0.345, 0.345, 0.345}
\newcommand{\hlnum}[1]{\textcolor[rgb]{0.686,0.059,0.569}{#1}}%
\newcommand{\hlstr}[1]{\textcolor[rgb]{0.192,0.494,0.8}{#1}}%
\newcommand{\hlcom}[1]{\textcolor[rgb]{0.678,0.584,0.686}{\textit{#1}}}%
\newcommand{\hlopt}[1]{\textcolor[rgb]{0,0,0}{#1}}%
\newcommand{\hlstd}[1]{\textcolor[rgb]{0.345,0.345,0.345}{#1}}%
\newcommand{\hlkwa}[1]{\textcolor[rgb]{0.161,0.373,0.58}{\textbf{#1}}}%
\newcommand{\hlkwb}[1]{\textcolor[rgb]{0.69,0.353,0.396}{#1}}%
\newcommand{\hlkwc}[1]{\textcolor[rgb]{0.333,0.667,0.333}{#1}}%
\newcommand{\hlkwd}[1]{\textcolor[rgb]{0.737,0.353,0.396}{\textbf{#1}}}%

\usepackage{framed}
\makeatletter
\newenvironment{kframe}{%
 \def\at@end@of@kframe{}%
 \ifinner\ifhmode%
  \def\at@end@of@kframe{\end{minipage}}%
  \begin{minipage}{\columnwidth}%
 \fi\fi%
 \def\FrameCommand##1{\hskip\@totalleftmargin \hskip-\fboxsep
 \colorbox{shadecolor}{##1}\hskip-\fboxsep
     % There is no \\@totalrightmargin, so:
     \hskip-\linewidth \hskip-\@totalleftmargin \hskip\columnwidth}%
 \MakeFramed {\advance\hsize-\width
   \@totalleftmargin\z@ \linewidth\hsize
   \@setminipage}}%
 {\par\unskip\endMakeFramed%
 \at@end@of@kframe}
\makeatother

\definecolor{shadecolor}{rgb}{.97, .97, .97}
\definecolor{messagecolor}{rgb}{0, 0, 0}
\definecolor{warningcolor}{rgb}{1, 0, 1}
\definecolor{errorcolor}{rgb}{1, 0, 0}
\newenvironment{knitrout}{}{} % an empty environment to be redefined in TeX

\usepackage{alltt}

%  Set theme (a nice plain one)
\usetheme{Malmoe}

%	Use named colors, set main color of theme
%		to match Web site color:
\definecolor{MainColor}{RGB}{10, 74, 109}
\colorlet{MainColorMedium}{MainColor!50}
\colorlet{MainColorLight}{MainColor!20}
\usecolortheme[named=MainColor]{structure} 


%	customize footline to have slide numbers
\defbeamertemplate*{footline}{my footline}
{
    \ifnum \insertpagenumber=1
      \leavevmode%
      \vskip0pt%
    \else
      \leavevmode%
      \hbox{%      
        \begin{beamercolorbox}[wd=.98\paperwidth,ht=.2ex,dp=1ex,right]{}%
      \insertframenumber/\inserttotalframenumber
      % \rjust \usebeamerfont{title in head/foot}a \href{http://statsteachr.org}{statsTeachR} resource
      \end{beamercolorbox}
      }%
      \vskip0pt%
    \fi
}

%	Get rid of navigation buttons
\setbeamertemplate{navigation symbols}{}

%	Make footnotes not ugly
\usepackage{hanging}
\setbeamertemplate{footnote}{\raggedright\hangpara{1em}{1}\makebox[1em][l]{\insertfootnotemark}\footnotesize\insertfootnotetext\par}


%        ******	Define title page	**********************
\setbeamertemplate{title page}{


% 	% Go to a lot of trouble to get the title in a
% 	%	nice box, since customizing a beamer block
% 	%	does not entirely work here (I don't know why)
	\newlength{\titleBoxWidth}
	\setlength{\titleBoxWidth}{\textwidth}
	\addtolength{\titleBoxWidth}{-2.0em}
	\setlength{\fboxsep}{1.0em}
	\setlength{\fboxrule}{0pt}
        {\color{MainColor}
			\raggedright
			\LARGE\textbf{\inserttitle}
        } % end color

	\vfill


	{a \href{http://statsteachr.org}{\bf{statsTeachR}} resource}

	\vspace{\baselineskip}
        
        \scriptsize{\LicenseText}


		\vspace{-15em}

	\clearpage
}	% end define title page

%  The following variables are assumed by the standard preamble:
%	Global variable containing module name:
\title{Principles and Practice of Reproducible Research with R}

%        Global variable containing text of authorship acknowledgments and license terms:
\newcommand{\LicenseText}{These slides were adapted for \href{http://statsteachr.org}{statsTeachR} by Emily Ramos from slides written by Andrea Foulkes, Gregory Matthews, Nicholas Reich and are released under a \href{http://creativecommons.org/licenses/by-sa/3.0/deed.en_US}{Creative Commons Attribution-ShareAlike 3.0 Unported License}. }



%	******	Document body begins here	**********************
\IfFileExists{upquote.sty}{\usepackage{upquote}}{}
\begin{document}

%	Title page
\begin{frame}[plain]
	\titlepage
\end{frame}

%	******	Everything through the above line must be placed at
%		the top of any TeX file using the statsTeachR standard
%		beamer preamble. 


\begin{frame}{}
\huge {\sc Biostatistics in Practice (BiP) 2014}: \\

\normalsize 
\bigskip
Data Sciences for the Life Sciences in a High Performance Computing Environment

\bigskip
\scriptsize \emph{Sponsored by the {\bf Graduate Program in Biostatistics at UMass Amherst}, the UMass Institute for Computational Biology, Biostatistics and Bioinformatics (ICB3) and the Massachusetts Green High Performance Computing Center (MGHPCC).}
\end{frame}


\begin{frame}{Agenda}

\bigskip
\textbf{Agenda}
\bigskip
\scriptsize{
\begin{enumerate}
\item[] \underline{\sc Module 1}: \hfill 9:00AM --10:15AM

{\bf Principles of Reproducible Research}
  \begin{itemize}
	\item \scriptsize Brief conceptual overview; Git/GitHub; RMarkdown
	\end{itemize}
\bigskip
\item[] \underline{\sc Module 2}: \hfill 10:45AM -- Noon

{\bf Simulation and Parallel Computing} 
	\begin{itemize}
	\item \scriptsize Simulation example with simple linear regression; Performing a permutation test; Running permutation tests in parallel
	\end{itemize}

\bigskip
\item[] \underline{\sc Module 3}:  \hfill 1:45PM -- 3:00PM

{\bf Introduction to Cluster Computing}
\begin{itemize}
\item \scriptsize VPN and ssh to MGHPCC; Data/code transfer; Parallel vs cluster computing; Submitting jobs (single and distributed)
\end{itemize}

\end{itemize}

\end{enumerate}
}\end{frame}



\begin{frame}{Special Thanks...}

\begin{block}{Individuals making this possible}
\begin{itemize}
\item John Goodhue
\item Claire Christopherson
\item Karen Utgoff
\item Ralph Zottola
\item Al Ritacco \& Chris Hull
\item UMass OIT 
\begin{itemize}
\item Chris Misra
\item Jim Mileski
\item Jason Houghton
\item Genti Lagji
\end{itemize}
\end{itemize}
\end{block}
\end{frame}


\begin{frame}{Special Thanks...}

\begin{block}{Support provided by:}
\begin{itemize}
\item President's office S\&T Initiative
\item Vice Chancellor Research \& Engagement
\item UMass College of Natural Sciences
\item UMass School of Public Health and Health Sciences
\item MGHPCC
\item UMass Graduate Program in Biostatistics
\end{itemize}
\end{block}
\end{frame}

\begin{frame}{Introductions}
Faculty:
\begin{itemize}
\item Dr. Nicholas Reich, Assistant Professor of Biostatistics, UMass Amherst
\item Dr. Gregory Mathews, Lecturer in Biostatistics, UMass Amherst
\item Dr. Andrea Foulkes, Associate Professor and Head of Biostatistics, UMass Amherst
\end{itemize}

\bigskip
Teaching Assistants:
\begin{itemize}
\item Steven Calderbank
\item Eric Cohen
\item Stephen Lauer
\item Sara Nu\~nez
\item Emily Ramos
\item Eric Reed
\end{itemize}
\end{frame}


\begin{frame}{Where we are coming from}

\begin{itemize}
        \item UMass-Amherst
                \begin{itemize}
                        \item Biostatistics
                        \item Mathematics and Statistics
                        \item Linguistics
                        \item Landscape Ecology
                        \item Veterinary and Animal Sciences Department
                        \item Political Science
                        \item Resource Economics
                        \item Biology
                        \item Organismic and Evolutionary Biology
                        \item Environmental Conservation
                \end{itemize}                
        \item UMass Medical School: Quantitative Health Sciences
        \item Babson College
        \item Northeastern University
        \item Framingham Heart Study
        \item Boston University
\end{itemize}

\end{frame}


\begin{frame}{Diverse background in R}

\begin{figure}[t]
    \includegraphics[width=\textwidth]{figures/Rbackground1}  
\end{figure}
\begin{figure}[t]
    \includegraphics[width=\textwidth]{figures/Rbackground2}  
\end{figure}

\end{frame}


\begin{frame}{Principles of Reproducible Research -- Definition}
\emph{Reproducible research has been defined in the scientific community as published scientific work that can be recreated using code and data made available by the authors:} 
\begin{itemize}
\item Creating reproducible research requires authors to carefully document approaches used to process, manage, analyze, and visualize data. 
\item It also requires authors to have a foundational understanding of the uncertainty that underlies the statistical model they use to describe their data. 
\end{itemize}
\end{frame}

\begin{frame}{Principles of Reproducible Research -- A Brief History}
\begin{itemize}
\item \scriptsize Roots of reproducible research can be traced to the concept of “literate programming” heralded by Donald Knuth \\
{\tiny -- Knuth, D. E. (1992). Literate Programming (1st ed.). Center for the Study of Language and Information.}
\item \scriptsize Concept  operationalized in 2002 by Friederic Leisch with introduction of Sweave, a program that allows the user to weave together R code and natural language descriptions \\
{\tiny -- Leisch, F. (2002a). Sweave. Dynamic generation of statistical reports using literate data analysis. SFB Adaptive Information Systems and Modelling in Economics and Management Science, WU Vienna University of Economics and Business; Leisch, F. (2002b). Sweave, part I: Mixing R and LaTeX. R News, 2/3, 28–31.
}
\item \scriptsize Importance of reproducibility discussed in vast array of fields, from econometrics, epidemiology and biostatistics, bioinformatics, and engineering \\
{\tiny -- Koenker, R. (1996). Reproducible econometric research. Retrieved September 17, 2012, from: \url{http://www.econ.uiuc.edu/~roger/research/repro/repro.html}; Peng, R. D. (2009). Reproducible research and Biostatistics. Biostatistics, 10(3), 405–408. doi:10.1093/biostatistics/kxp014; Gentleman, R. (2005). Reproducible research: a bioinformatics case study. Statistical applications in genetics and molecular biology, 4, Article2. doi:10.2202/1544-6115.1034; Vandewalle, P., Barrenetxea, G., Jovanovic, I., Ridolfi, A., \& Vetterli, M. (2007). Experiences with Reproducible Research in Various Facets of Signal Processing Research. IEEE International Conference on Acoustics, Speech and Signal Processing. Proceedings, 4, IV–1256. doi:10.1109/ICASSP.2007.367304)}
\end{itemize} 
\end{frame}

\begin{frame}{Principles of Reproducible Research -- Training}
Foundational training in reproducible research includes rigorous instruction in:
\begin{enumerate}
\item Analysis and visualization of data through literate programming (Module 1 \& 2);
\item Dissemination of open-source and extensible code, documentation, and, whenever possible, data \\ (Modules 1 \& 2);
\item Basic statistical literacy, including characterization of uncertainty using sound statistical and computational principles (Modules 2 \& 3).
\end{enumerate}
\end{frame}


\begin{frame}{Tools for reproducible data analysis with R}

\begin{block}{Topics}
\begin{itemize}
        \item \textbf{R/RStudio}
       \item Version control: git \& GitHub.com
         \item ggplot2
        \item Dynamic documents: knitr, RMarkdown, Sweave
 \end{itemize}
\end{block}

\end{frame}



%Opening R - some basics (getwd setwd)
%Data Management
\begin{frame}{Introduction to R}
What is R?
\begin{itemize}
\item R is a language and environment for statistical computing and graphics. 
\item Website: www.r-project.org
\end{itemize}
\end{frame}

%%%%%%%%%%%%%
\begin{frame}{Introduction to R}
Pros
\begin{itemize}
\item R is free.
\item There are many packages for R.  Chances are someone has written it already. 
\item Many cutting edge techniques are available very quickly.  
\end{itemize}
\end{frame}

%%%%%%%%%%%%%

\begin{frame}{Introduction to R}
Cons
\begin{itemize}
\item R is free.  This means there is no support for R.  
\item This also means that you use a package at your own risk and you trust the author wrote the code correctly.  
\item Takes a little while to learn.
\end{itemize}
\end{frame}

%%%%%%%%%%%%%


%%%%%%%%%%%%%
%\begin{frame}
%\includegraphics[scale=.54]{twitterMap.pdf}
%\end{frame}
\begin{frame}
\frametitle{RStudio}
What is Rstudio$^{TM}$?
\begin{itemize}
\item RStudio? is a free and open source integrated development environment (IDE) for R.
\item rstudio.org
\end{itemize}
\end{frame}

%%%%%%%%%%%%%

\begin{frame}
\frametitle{RStudio}

Demo...

\end{frame}



%%%%%%%%%%%%%



\begin{frame}{Tools for reproducible data analysis with R}

\begin{block}{Topics}
\begin{itemize}
        \item R/RStudio
       \item \textbf{Version control: git \& GitHub.com}
         \item ggplot2
        \item Dynamic documents: knitr, RMarkdown, Sweave
 \end{itemize}
\end{block}

\end{frame}


\begin{frame}{Version control systems}

\begin{block}{Common VCS}
\begin{itemize}
\item git
\item subversion (svn)
\item mercurial
\item ...
\end{itemize}
\end{block}

\end{frame}


\begin{frame}{Version control and reproducibility}


\begin{block}{Why version control?}

\begin{itemize}
        \item allows you to roll back to previous versions easily
        \item allows you to try things out without disrupting existing code
        \item versions can be flagged as ``releases''  
\end{itemize}
\end{block}

\end{frame}

\begin{frame}{Version control systems}

\begin{figure}[t]
    \includegraphics[width=\textwidth]{figures/VCS/VCS001.jpg}  
\end{figure}

\end{frame}

\begin{frame}{Version control systems (git flavored)}

\begin{figure}[t]
    \includegraphics[width=\textwidth]{figures/VCS/VCS002.jpg}  
\end{figure}

\end{frame}

\begin{frame}{Version control systems}

\begin{figure}[t]
    \includegraphics[width=\textwidth]{figures/VCS/VCS003.jpg}  
\end{figure}

\end{frame}

\begin{frame}{Version control systems  (git/GitHub flavored)}

\begin{figure}[t]
    \includegraphics[width=\textwidth]{figures/VCS/VCS004.jpg}  
\end{figure}

\end{frame}


\begin{frame}{Lots of (mostly free) options for cloud-based version controlling}

\begin{block}{Most services host multiple types of VCS}
\begin{itemize}
\item sourceforge.net
\item github.com
\item bitbucket.org
\item springloops.io
\item ... [what have you used?]
\end{itemize}
\end{block}

\end{frame}



\begin{frame}{git is a dialect}

\begin{block}{Key command-line operations}
\begin{itemize}
\item {\tt git init}: initializes a repository locally
\item {\tt git clone}: clones a repository from a remote source (i.e. GitHub.com)
\item {\tt git branch}: creates a new ``branch'' of code
\item {\tt git add}, {\tt git rm}: manipulating files 
\item {\tt git commit}: commits changes you have made
\end{itemize}
\end{block}

\end{frame}


\begin{frame}{Using git with RStudio}

\begin{block}{Demo... }
\begin{itemize}
        \item clone a repository from GitHub
        \item simple commit/push/pull examples
\end{itemize}
\end{block}

\end{frame}


\begin{frame}{Tools for reproducible data analysis with R}

\begin{block}{Topics}
\begin{itemize}
        \item R/RStudio
       \item Version control: git \& GitHub.com
         \item \textbf{ggplot2}
        \item Dynamic documents: knitr, RMarkdown, Sweave
 \end{itemize}
\end{block}

\end{frame}


\begin{frame}{Arguments for using {\tt ggplot2}}

\begin{block}{From {\tt ggplot2} users}
\begin{itemize}
\item ``literate programming: you describe the plot almost in a natural language. It makes it much easier to reuse code and come back to a graph in a few years time without feeling lost.''
\item ``flexibility, intuitiveness, and logic of the mapping between the data and its representation. Once one gets his mind wrapped around the grammar of graphics concepts (and particularly the aesthetic mapping), figuring out how to best represent a dataset is much easier than with other graphical representation methods.'' 
\item ``I think the big strength of ggplot2 is graphic-artist quality default output. And an information-focussed point of view.''
\item \href{https://github.com/hadley/ggplot2/wiki/Why-use-ggplot2}{Lots more reasons here..}
\end{itemize}
\end{block}

Today, we will be using very simple {\tt ggplot2} commands.

\end{frame}

\begin{frame}{Tools for reproducible data analysis with R}

\begin{block}{Topics}
\begin{itemize}
        \item R/RStudio
       \item Version control: git \& GitHub.com
         \item ggplot2
        \item \textbf{Dynamic documents: knitr, RMarkdown, Sweave}
 \end{itemize}
\end{block}

\end{frame}


\begin{frame}{Dyanamic Documents in R}
\begin{itemize}
\item Dynamic R documents allow a user to combine text, R code, and R output, including tables and figures, into one document.
\item Why is this useful? 
\begin{itemize}
\item Writing code and producing reports are now one document rather than many.
\item When the analysis changes, the results in the report change automatically.  
\item What else?
\end{itemize}
\item There are several options for how to do this.  
\begin{itemize}
\item R Markdown
\item Sweave
\item knitr
\end{itemize}
\end{itemize}
%``dynamic documents''
%R markdown
%Sweave
%knitr
\end{frame}


\begin{frame}{Dyanamic R Reports: Summary}
\begin{itemize}
\item R Markdown: creates HTML document
\item Sweave: creates pdf document AND incorporates LaTeX
\item knitr: $\approx$ Sweave + cacheSweave + pgfSweave + weaver + animation::saveLatex + R2HTML::RweaveHTML + highlight::HighlightWeaveLatex + 0.2 * brew + 0.1 * SweaveListingUtils + more
\end{itemize}
%``dynamic documents''
%R markdown
%Sweave
%knitr
\end{frame}

\begin{frame}{R Markdown}
\begin{itemize}
\item R Markdown creates HTML files  
\item Reference: http://www.rstudio.com/ide/docs/authoring/using\_markdown
\item Markdown files (.Rmd) act just like text files, except they allow a user to embed R code in chunks
\item The syntax for a chunk in R Markdown:
\end{itemize}
Regular text \\
```\{r\}\\
Code goes here\\
```\\
\end{frame}

\begin{frame}{Sweave}
\begin{itemize}
\item Sweave creates pdf files as output
\item Reference: http://leisch.userweb.mwn.de/Sweave/
\item Sweave Manual: http://www.stat.uni-muenchen.de/~leisch/Sweave/Sweave-manual.pdf
\item Sweave not only integrates R code, but also LaTeX!
\item The syntax for a chunk in Sweave:
\end{itemize}
Regular text with \$LaTeX\$ if you want it.\\
\textless\textless OPTIONS \textgreater\textgreater=\\
Code goes here\\
@\\
\end{frame}

\begin{frame}{Sweave: Options}
\begin{itemize}
\item fig=TRUE (or FALSE): This indicates that the code in the chunk will print the figure to the output pdf document
\item echo=TRUE (or FALSE): Should the R input code be displayed in the output pdf document
\item eval=TRUE (or FALSE): Should the R input code be evaluated
\end{itemize}
%Split and label?
%reuse code chunks?
\end{frame}



\begin{frame}{knitr}
\begin{itemize}
\item Created by Ph.D. student Yihui Xie (what have you done?) 
\item knitr creates pdf files as output.
\item It also allows the use of LaTeX, like Sweave, whereas R Markown does not.  
\item Syntax for knitr is largely the same as Sweave.
\item Xie describes knitr $\approx$ Sweave + cacheSweave + pgfSweave + weaver + animation::saveLatex + R2HTML::RweaveHTML + highlight::HighlightWeaveLatex + 0.2 * brew + 0.1 * SweaveListingUtils + more
\end{itemize}
\end{frame}




%%%%%%%%%%%%%%%%%%%%%%%%%%%%%%%%%%%%%%%%%%%%%%%%%%%%%%%
%%% TEMPLATES FOR STATSTEACHR FORMATTING BELOW HERE  %%
%%%%%%%%%%%%%%%%%%%%%%%%%%%%%%%%%%%%%%%%%%%%%%%%%%%%%%%

% \begin{frame}{Overview}
% 
% 	\begin{block}{Diagnostic Test}
% 		Tool for deciding on something you want to know --- the true status of some outcome --- based on something you can measure.
% 
% 		\begin{itemize}
% 
% 			\item{Deciding whether someone has tuberculosis (the outcome) based on the presence or size of a raised swelling (the test) following tuberculin injection.}
% 
% 			\item{Deciding whether an infant has phenylketonuria (the outcome) based on the blood level of phenylalanine (the test).}
% 
% 			\item{Deciding whether an infant has phenylketonuria (the outcome) based on the odor of the urine (same outcome of interest, different test).}
% 		\end{itemize}
% 
% 	\end{block}
% 
% \end{frame}
% 
% 
% 
% \begin{frame}{The Gold Standard}
% 
% In evaluating a test, must compare results to some ``truth''.
% 
% This is provided by the best possible test --- the \emph{gold standard} or \emph{criterion standard} test.
% 
% \medskip
% 
% 	\begin{block}{Why Not Use the Gold Standard?}
% 
% 		 May be expensive, invasive, or impractical.
% 
% 		\begin{itemize}
% 
% 			\item{Gold-standard test for pancreatic cancer involves biopsy of the pancreas;}
% 			\item{Gold-standard test for Alzheimer's disease is post-mortem brain autopsy.}
% 
% 		\end{itemize}
% 
% 	\end{block}
% 
% \end{frame}
% 
% 
% 
% \begin{frame}{Sensitivity and Specificity}
% 
% \taburulecolor{MainColorMedium}
% \tabulinesep=1.5ex
% \begin{tabu} to 1.0\textwidth[r]{X[,L,]X[,L,]}
% 	\rowcolor{MainColorLight}\strong{Sensitivity}                                                                    	&
% 	\strong{Specificity}
% \\
% 	How good is this test at \textit{detecting cases} of this condition?
% 	&
% 	How good is this test at giving a \textit{clean bill of health} to those \textit{without} the condition?
% \\\hline
% 	What proportion, of those with the condition, test positive for the condition?
% 	&
% 	What proportion, of those without the condition, test negative for the condition?
% \\\hline
% 	\vspace{-3ex}\begin{displaymath}\frac{true\ positives}{positives\ in\ population}\end{displaymath}\vspace{-3ex}
% 	&
% 	\vspace{-3ex}\begin{displaymath}\frac{true\ negatives}{negatives\ in\ population}\end{displaymath}\vspace{-3ex}
% \\\hline
% 	\begin{math}
% 		Pr(T^{+} \mid D^{+})
% 	\end{math}
% 	\newline
% 	$T^{+}$ --- positive test;\newline
% 	$D^{+}$ --- disease
% 	&
% 	\begin{math}
% 		Pr(T^{-} \mid D^{-})
% 	\end{math}
% 	\newline
% 	$T^{-}$ --- negative test;\newline
% 	$D^{+}$ --- no disease
% \\
% \end{tabu}
% 
% \end{frame}


\end{document}
